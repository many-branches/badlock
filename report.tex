\documentclass[conference]{IEEEtran}
\IEEEoverridecommandlockouts
% The preceding line is only needed to identify funding in the first footnote. If that is unneeded, please comment it out.
\usepackage{cite}
\usepackage{amsmath,amssymb,amsfonts}
\usepackage{algorithmic}
\usepackage{graphicx}
\usepackage{textcomp}
\usepackage{xcolor}
\def\BibTeX{{\rm B\kern-.05em{\sc i\kern-.025em b}\kern-.08em
    T\kern-.1667em\lower.7ex\hbox{E}\kern-.125emX}}
\begin{document}

\title{Badlock: Static Reentrant Deadlock Detection for Rust via Taint Analysis}

\author{\IEEEauthorblockN{Liam Monninger}
\IEEEauthorblockA{\textit{Department of Computer and Information Science, SEAS} \\
\textit{University of Pennsylvania}\\
Philadelphia, PA, USA \\}}

\maketitle

\begin{abstract}
The badlock
\end{abstract}

\begin{IEEEkeywords}
static analysis, deadlocks, rust, taint analysis
\end{IEEEkeywords}

\section{Introduction}

\subsection{Taint Analysis}

\subsection{Rust std::sync}

\section{Badlock}

\subsection{Datalog Program}

\subsection{Evaluation}
We are currently capable of detecting deadlocks in the following simple programs. Programs of this form were defined as the original POC.

\subsubsection{No Deadlock}
We were able to successfully analyzing clearly non-deadlocking Rust programs wherein no lock is ever acquired. The below is an example of a passing program.

\begin{verbatim}
fn main(){

    let a = 0;
    for i in 0..10 {
        let b = a + i;
        println!("{}", b);
    }
    
}
\end{verbatim}

\subsubsection{Simple Deadlock}
We successfully analyzed simple deadlocking programs which use std::sync::Mutex such as the following.

\begin{verbatim}
use std::sync::Mutex;

fn main() {

    let mut safe_x = Mutex::new(64);
    
    let mut guard = safe_x.lock().unwrap();
    *guard += 1;
    println!("Here is fine x: {}", *guard);

    let mut deadlock = safe_x.lock().unwrap();
    *deadlock += 1;
    println!("Should never get here x: {}", *deadlock);

}
\end{verbatim}

\subsubsection{Appropriate Lock Acquisition}
We successfully analyzed simple programs where locks are acquired and locked in a manner preventing reentrancy. Notably, we successfully analyzed the implicit lock releases that occur when a Rust lifetime ends.

\begin{verbatim}
use std::sync::Mutex;

fn main() {

    let mut x = 64;
    let mut safe_x = Mutex::new(x);
    
    {
        let mut guard = safe_x.lock().unwrap();
        *guard += 1;
        println!("Here is fine x: {}", *guard);
    }

    {
        let mut no_deadlock = safe_x.lock().unwrap();
        *no_deadlock += 1;
        println!("Should also get here x: {}", *no_deadlock);
    }

}
\end{verbatim}

\section{Conclusion}

\begin{thebibliography}{00}

\end{thebibliography}


\end{document}
